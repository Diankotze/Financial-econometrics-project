\documentclass[11pt,preprint, authoryear]{elsarticle}

\usepackage{lmodern}
%%%% My spacing
\usepackage{setspace}
\setstretch{1.2}
\DeclareMathSizes{12}{14}{10}{10}

% Wrap around which gives all figures included the [H] command, or places it "here". This can be tedious to code in Rmarkdown.
\usepackage{float}
\let\origfigure\figure
\let\endorigfigure\endfigure
\renewenvironment{figure}[1][2] {
    \expandafter\origfigure\expandafter[H]
} {
    \endorigfigure
}

\let\origtable\table
\let\endorigtable\endtable
\renewenvironment{table}[1][2] {
    \expandafter\origtable\expandafter[H]
} {
    \endorigtable
}


\usepackage{ifxetex,ifluatex}
\usepackage{fixltx2e} % provides \textsubscript
\ifnum 0\ifxetex 1\fi\ifluatex 1\fi=0 % if pdftex
  \usepackage[T1]{fontenc}
  \usepackage[utf8]{inputenc}
\else % if luatex or xelatex
  \ifxetex
    \usepackage{mathspec}
    \usepackage{xltxtra,xunicode}
  \else
    \usepackage{fontspec}
  \fi
  \defaultfontfeatures{Mapping=tex-text,Scale=MatchLowercase}
  \newcommand{\euro}{€}
\fi

\usepackage{amssymb, amsmath, amsthm, amsfonts}

\def\bibsection{\section*{References}} %%% Make "References" appear before bibliography


\usepackage[round]{natbib}
\bibliographystyle{plainnat}

\usepackage{longtable}
\usepackage[margin=2cm,bottom=2cm,top=2.5cm, includefoot]{geometry}
\usepackage{fancyhdr}
\usepackage[bottom, hang, flushmargin]{footmisc}
\usepackage{graphicx}
\numberwithin{equation}{section}
\numberwithin{figure}{section}
\numberwithin{table}{section}
\setlength{\parindent}{0cm}
\setlength{\parskip}{1.3ex plus 0.5ex minus 0.3ex}
\usepackage{textcomp}
\renewcommand{\headrulewidth}{0.2pt}
\renewcommand{\footrulewidth}{0.3pt}

\usepackage{array}
\newcolumntype{x}[1]{>{\centering\arraybackslash\hspace{0pt}}p{#1}}

%%%%  Remove the "preprint submitted to" part. Don't worry about this either, it just looks better without it:
\makeatletter
\def\ps@pprintTitle{%
  \let\@oddhead\@empty
  \let\@evenhead\@empty
  \let\@oddfoot\@empty
  \let\@evenfoot\@oddfoot
}
\makeatother

 \def\tightlist{} % This allows for subbullets!

\usepackage{hyperref}
\hypersetup{breaklinks=true,
            bookmarks=true,
            colorlinks=true,
            citecolor=blue,
            urlcolor=blue,
            linkcolor=blue,
            pdfborder={0 0 0}}


% The following packages allow huxtable to work:
\usepackage{siunitx}
\usepackage{multirow}
\usepackage{hhline}
\usepackage{calc}
\usepackage{tabularx}
\usepackage{booktabs}
\usepackage{caption}
\usepackage{colortbl}

\urlstyle{same}  % don't use monospace font for urls
\setlength{\parindent}{0pt}
\setlength{\parskip}{6pt plus 2pt minus 1pt}
\setlength{\emergencystretch}{3em}  % prevent overfull lines
\setcounter{secnumdepth}{5}

%%% Use protect on footnotes to avoid problems with footnotes in titles
\let\rmarkdownfootnote\footnote%
\def\footnote{\protect\rmarkdownfootnote}
\IfFileExists{upquote.sty}{\usepackage{upquote}}{}

%%% Include extra packages specified by user
% Insert custom packages here as follows
% \usepackage{tikz}

%%% Hard setting column skips for reports - this ensures greater consistency and control over the length settings in the document.
%% page layout
%% paragraphs
\setlength{\baselineskip}{12pt plus 0pt minus 0pt}
\setlength{\parskip}{12pt plus 0pt minus 0pt}
\setlength{\parindent}{0pt plus 0pt minus 0pt}
%% floats
\setlength{\floatsep}{12pt plus 0 pt minus 0pt}
\setlength{\textfloatsep}{20pt plus 0pt minus 0pt}
\setlength{\intextsep}{14pt plus 0pt minus 0pt}
\setlength{\dbltextfloatsep}{20pt plus 0pt minus 0pt}
\setlength{\dblfloatsep}{14pt plus 0pt minus 0pt}
%% maths
\setlength{\abovedisplayskip}{12pt plus 0pt minus 0pt}
\setlength{\belowdisplayskip}{12pt plus 0pt minus 0pt}
%% lists
\setlength{\topsep}{10pt plus 0pt minus 0pt}
\setlength{\partopsep}{3pt plus 0pt minus 0pt}
\setlength{\itemsep}{5pt plus 0pt minus 0pt}
\setlength{\labelsep}{8mm plus 0mm minus 0mm}
\setlength{\parsep}{\the\parskip}
\setlength{\listparindent}{\the\parindent}
%% verbatim
\setlength{\fboxsep}{5pt plus 0pt minus 0pt}



\begin{document}

\begin{frontmatter}  %

\title{Correlation between bond returns' in developed economies: Evidence from
6 developed countries.}

% Set to FALSE if wanting to remove title (for submission)




\author[Add1]{Dian Kotze}
\ead{kotzedian11@gmail.com}





\address[Add1]{Stellenbosch University, Stellenbosch, South Africa}


\begin{abstract}
\small{
This paper will compare the changing time-varying correlations of 6
developed countries' 10-year bond returns over two time periods. The
changing levels of time-varying correlations between the 10-year bond
yields of Australia, Canada, Korea, New Zealand, the UK and the US will
be analysed and compared. The study looks at the period prior to the
global financial crisis and the period post-crisis. The co-movement
between the bond markets is of particular interest to investors who seek
to diversify their portfolios. Whilst many multi-variate models are
available, this paper applies the Dynamic Conditional Correlation (DCC)
Multivariate Generalized Autoregressive Conditional Heteroskedasticity
(MV-GARCH) modelling technique to estimate time varying conditional
correlations. This will allow for the comparison of the returns of these
bonds and how the correlation changed post global financial crisis. The
paper finds high correlation between all the country pairs, in excess of
60 and 50 percent of the pre-crisis and post-crisis periods
respectively. The only exception being Korea, whose correlation with all
countries is low, but increases in the period after the global financial
crisis. This portfolio offers little, if any diversification potential
and is therefore not a good portfolio to hold in isolation.
}
\end{abstract}

\vspace{1cm}

\begin{keyword}
\footnotesize{
Multivariate GARCH \sep DCC \sep Bonds \sep Developed economies \\ \vspace{0.3cm}
\textit{JEL classification} L250 \sep L100
}
\end{keyword}
\vspace{0.5cm}
\end{frontmatter}



%________________________
% Header and Footers
%%%%%%%%%%%%%%%%%%%%%%%%%%%%%%%%%
\pagestyle{fancy}
\chead{}
\rhead{Financial Econometrics 871}
\lfoot{}
\rfoot{\footnotesize Page \thepage\\}
\lhead{}
%\rfoot{\footnotesize Page \thepage\ } % "e.g. Page 2"
\cfoot{}

%\setlength\headheight{30pt}
%%%%%%%%%%%%%%%%%%%%%%%%%%%%%%%%%
%________________________

\headsep 35pt % So that header does not go over title




\section{\texorpdfstring{Introduction
\label{Introduction}}{Introduction }}\label{introduction}

The global financial markets are very turbulent even at the best of
times. Bonds offer a safer alternative to stocks and equities,
naturally, with a significantly lower return. Intuitively, the greater
the correlation or co-movement between the returns of the bonds, the
less diversification potential. Therefore, it is important to identify
the correlation between the bonds returns, in order to generate a
portfolio which could potentially provide good returns while minimising
downside risk. Ideally, correlation between the assets should be
relatively low. This enables spreading of risk of the portfolio. The
main focus, however, is the correlation of the bond markets,
particularly developed economies' bond returns. The correlation between
developed economies' bond returns is expected to be high, given the
global integration of the financial markets and similar financial market
conditions (Nico Katzke \protect\hyperlink{ref-katzke2013}{2013}). The
paper will investigate and add insights to correlation amongst developed
economies' bond markets.

This paper aims to study the correlation between the weekly returns of
the 6 countries' 10-year bonds. Studying the periods before and after
the global financial crisis will enable comparison across periods to
identify effects and changes. This will also demonstrate the need for
diversification in a portfolio, which can reduce the risk of losses. The
paper will begin with a brief introduction to bonds and correlation in
the bonds market in developed economies. This will be followed by a
description of the data and the required transformations. The
methodology section will provide insight to the approach taken to obtain
the correlations. The results of the findings will then be reported and
will be followed by concluding remarks.

\section{\texorpdfstring{Literature review and rationale
\label{lit_review}}{Literature review and rationale }}\label{literature-review-and-rationale}

Bonds, in the simplest of terms, represents a debt obligation.
Effectively, the money that is received by the person/entity that issues
a bond, is a loan. Lenders require an incentive to provide their money
as a source of financing. The incentive comes in the form of interest
that is paid on these bonds, which ultimately attracts investors and is
referred to as the yield. Bonds offer investors a safe alternative to
stocks (equity), whose repayment is based on profitability (Fama and
French \protect\hyperlink{ref-FRENCH}{1989}). Whilst potential returns
are far greater than that of bonds, the potential losses in the case of
stocks can be far more severe. Essentially, bonds provide a low risk
alternative to investors who do not have an appetite for risk.

According to Cappiello, Engle, and Sheppard
(\protect\hyperlink{ref-cappiello2006}{2006}), portfolio management
requires two broad strategies. Firstly, to invest in different asset
classes which have little to negative correlation. Secondly, investing
in the same or similar asset markets, while diversifying through
investing in multiple international markets. This paper only considers
the second strategy and is secondary to the main objective of
identifying correlation between the developed economies bond returns.
Now that a brief understanding of what bonds are and the use of
diversification, lets gain an understanding of the correlation between
bond markets in developed economies.

Developed economies in general, do display relatively high correlation
with other developed economies, due in part, according to Barr and
Priestley (\protect\hyperlink{ref-barr2004}{2004}) to the growing
integration of the global financial markets. Additionally, the ease of
access to information in the 21st century, which means investors are
able to make real-time decisions which results in partially coordinated
actions (Kumar and Okimoto \protect\hyperlink{ref-kumar2011}{2011}).
Barr and Priestley (\protect\hyperlink{ref-barr2004}{2004}) finds
evidence of high correlation between the Canadian and US bond markets
and also finds correlation (although to a lesser degree) between the US
and UK bond markets. Moreover, Davies
(\protect\hyperlink{ref-davies2007}{2007}) notes in a study including 6
bond markets, that the US, UK and Canadian bond markets share a common
trend, subject to structural change. Davies
(\protect\hyperlink{ref-davies2007}{2007}) also finds these results for
some of the countries which are not included in this study and finds
support for the growing integration of the international bond market.
Lucey and Steeley (\protect\hyperlink{ref-lucey2006}{2006}) notes that
previous studies on the correlation of international bond markets are
few but increasing. Not many studies have included countries selected in
this study, as a result, this paper will add insights regarding the
correlation amongst developed countries.

This study aims to provide additional insights to the literature
regarding the correlation of the bond market, specifically that of
developed economies. While methods such as ADCC, and EWMA to name just
two of several models that can be used to model the conditional
time-varying variance-covariance are available, this paper, due to time
constraints will only discuss the DCC-GARCH methodology. This will be
used to briefly note the need for diversification.

\section{\texorpdfstring{Data \label{data}}{Data }}\label{data}

The data used in the paper is obtained from the Bloomberg terminals,
extracting the weekly returns of the 10-year bonds of 6 countries. The
countries included in the study are Australia (AUS), Canada, Korea,
United Kingdom (UK), New Zealand (NZ) and the United States of America
(US). The data returns are calculated as follows:

\begin{align} \label{eq:returns}
r_{i,t} = \frac{B_{i,t}}{B_{i,t-1}} - 1
\end{align}

The weekly returns as calculated according to equation \ref{eq:returns}
and are then log transformed, such that inferences regarding level
changes over time can be made. To clean the data, the paper uses Boudt's
technique. Additionally, the paper excludes the global financial crisis.
This provides the benefit of the crisis not influencing the results
dramatically and also allows for the evaluation of the impact of the
crisis on volatility and co-movement in these bond markets. Tables
\ref{tablestats} and \ref{tablestats2} show the mean returns and the
standard deviation of the returns. These appear to be the same for most
countries, due to the rounding of the values, which tend to show the
values are the same. However, it must be noted that this is an early
sign that the correlation between these countries' bond returns may be
high. The period prior to the crisis starts in the first week of
February and ends in the final week of December 2009. The post-crisis
period starts in the first week of October 2009 and ends in the final
week of June 2018.\footnote{Tables \ref{tablestats} and
  \ref{tablestats2} include 4 decimal places to illustrate that the
  means and standard deviations differ by country.}

\begin{longtable}{rlrr}
  \hline
 & Country & mean\_returns & std\_dev\_returns \\ 
  \hline
1 & AUS & 0.0906 & 0.0206 \\ 
  2 & Canada & 0.0895 & 0.0188 \\ 
  3 & Korea & 0.0900 & 0.0229 \\ 
  4 & NZ & 0.0903 & 0.0162 \\ 
  5 & UK & 0.0900 & 0.0178 \\ 
  6 & US & 0.0896 & 0.0258 \\ 
   \hline
\hline
\caption{Summary statistics: Bond Yields \label{tablestats}} 
\end{longtable}

\begin{longtable}{rlrr}
  \hline
 & Country & mean\_returns & std\_dev\_returns \\ 
  \hline
1 & AUS & 0.0906 & 0.0206 \\ 
  2 & Canada & 0.0895 & 0.0188 \\ 
  3 & Korea & 0.0900 & 0.0229 \\ 
  4 & NZ & 0.0903 & 0.0162 \\ 
  5 & UK & 0.0900 & 0.0178 \\ 
  6 & US & 0.0896 & 0.0258 \\ 
   \hline
\hline
\caption{Summary statistics post crisis: Bond Yields \label{tablestats2}} 
\end{longtable}

\section{\texorpdfstring{Methodology
\label{methodology}}{Methodology }}\label{methodology}

The concept of multivariate modelling involves mapping multiple
univariate GARCH processes to the multivariate domain, whereby a series
of n Univariate GARCH models are estimated. The paper will apply the
Dynamic Conditional Correlation model, first proposed by R. Engle
(\protect\hyperlink{ref-engle2002dynamic}{2002}). Essentially this is a
two-step process. Firstly, generalising the univariate GARCH processes
to the multivariate domain. Then using this to estimate the DCC model
which will allow for the estimation of the dynamic correlation estimates
for each series. This section will first explain how the univariate
GARCH processes can be generalised to the multivariate domain, followed
by a detailed description of the DCC methodology used to obtain the
dynamic correlation estimates.

The generalization is as follows: Given the stochastic process,
\(x_t, t=1,2,...T\) of financial returns with dimension \(N \times 1\)
and mean vector \(\mu_t\), given the information set \(I_{t-1}\):

\begin{align} \label{eq:mgarch1}
x_t \left| I_{t - 1} \right. = \mu_t + \varepsilon_t
\end{align}

where the residuals of the process are modelled as:

\begin{align} \label{eq:mgarch2}
\varepsilon_t = H_{t}^{1/2}z_t, 
\end{align}

\(H_t^{1/2}\) above is a \(N\times N\) positive definite matrix such
that \(\bf{H_t}\) is the conditional covariance matrix of
\(\bf{x_t}\).\footnote{\(\bf{z_t}\) is a \(N\times 1\) i.i.d. N(0,1)
  series}

The DCC-GARCH methodology which follows, will explain how this can be
used to map the \(H_t\) matrix into the multivariate plain.

\subsection{\texorpdfstring{DCC-GARCH methodology
\label{garch}}{DCC-GARCH methodology }}\label{dcc-garch-methodology}

DCC models offer a simple and more parsimonious means of doing MV-vol
modelling N.F. Katzke (\protect\hyperlink{ref-Texevier}{2017}). In
particular, it relaxes the constraint of a fixed correlation structure
assumed by the constant conditional correlation (CCC) model to allow for
estimates of time-varying correlation. The DCC model can be defined as:

\begin{equation} \label{dcc}
H_t = D_t.R_t.D_t.
\end{equation}

Equation \ref{dcc}, shows that \(H_t\), which as mentioned above is the
conditional covariance matrix of \(\bf{x_t}\), this splits the
variance-covariance matrix into identical diagonal matrices and an
estimate of the time-varying correlation \(R_T\). Estimating \(R_T\)
requires it to be inverted at each estimated period, therefore, a proxy
equation is used (R. Engle
\protect\hyperlink{ref-engle2002dynamic}{2002}):

\begin{align}  \label{dcc2}
Q_{ij,t} &= \bar Q + a\left(z_{t - 1}z'_{t - 1} - \bar{Q} \right) + b\left( Q_{ij, t - 1} - \bar{Q} \right) \hfill \\ \notag
                            &= (1 - a - b)\bar{Q} + az_{t - 1}z'_{t - 1} + b.Q_{ij, t - 1} \notag
                            \end{align}

Note the above equation is similar in form to a GARCH(1,1) process, with
non-negative scalars \(a\) and \(b\), and with: \(Q_{ij, t}\) the
unconditional sample variance estimate between series \(i\) and \(j\),
and lastly \(\bar{Q}\) the unconditional matrix of standardized
residuals from each univariate pair estimate. Using this information, we
can now to use equation \ref{dcc2} to estimate \(R_t\) as:

\begin{align}\label{eq:dcc3}
R_t &= diag(Q_t)^{-1/2}Q_t.diag(Q_t)^{-1/2}. 
\end{align}

Which contains the bivariate elements:

\begin{align}
R_t &= \rho_{ij,t} = \frac{q_{i,j,t}}{\sqrt{q_{ii,t}.q_{jj,t}}} 
\end{align}

This process produces the DCC model used to obtain the correlation
estimates and is formulated as:

\begin{align}
\varepsilon_t &\thicksim  N(0,D_t.R_t.D_t) \notag \\
D_t^2 &\thicksim \text{Univariate GARCH(1,1) processes $\forall$ (i,j), i $\ne$ j} \notag \\
z_t&=D_t^{-1}.\varepsilon_t \notag \\
Q_t &= \bar{Q}(1-a-b)+a(z_t'z_t)+b(Q_{t-1}) \notag \\
                        R_t &= Diag(Q_t^{-1}).Q_t.Diag({Q_t}^{-1}) \notag \\
                        \end{align}

\section{Emperical results}\label{emperical-results}

The findings of the bivariate correlations between the 6 countries will
be reported in this section. It is important to note that given the
selection of 6 developed economies, the correlation is expected to be
high as the financial conditions in the countries will be similar.
Firstly, this section looks at the volatility plots of the 6 countries
and how this changes between the period before the global financial
crisis and the period post-crisis. This will be followed by studying
selected bivariate correlation between the countries which display
interesting findings, and will be compared between the two periods
before and after the global financial crisis.

\subsection{\texorpdfstring{Volatility
\label{volatility}}{Volatility }}\label{volatility}

The volatility of the bond market before the global financial crisis is
significantly lower than after the crisis. The volatility of the returns
in the US is the highest for a large portion of the period before the
crisis, for as long as 2002-2005. Although it must be noted that this
doesn't increase significantly post crisis. The UK and Canadian returns
experience the greatest increase in volatility between the period before
and the period after the crisis. The return on the UK bond, post-crisis,
displays volatility of almost 6 times that of the volatility experienced
before the crisis, peaking in the final quarter of 2016. While the
volatility in the Canadian bond returns triples post-crisis compared to
before the crisis. It is evident from these volatility plots that the
correlation between the countries is high. According to Kumar and
Okimoto (\protect\hyperlink{ref-kumar2011}{2011}), there has been a
significant amount of uncertainty in the economy after the global
financial crisis. This adds to the explanation of the high and increased
volatility compared to the period before the crisis.

\begin{figure}[H]

{\centering \includegraphics{dian_finmetrics_project_files/figure-latex/figure1-1} 

}

\caption{Volatility plot \label{lit}}\label{fig:figure1}
\end{figure}

\begin{figure}[H]

{\centering \includegraphics{dian_finmetrics_project_files/figure-latex/figure2-1} 

}

\caption{Volatility plot post-crisis \label{lit}}\label{fig:figure2}
\end{figure}

\subsection{\texorpdfstring{Dynamic correlation
\label{corr}}{Dynamic correlation }}\label{dynamic-correlation}

The results of the Dynamic Conditional correlations show very high
correlation in almost all cases with all pairs. Figures \ref{figure3} to
\ref{figure8} show the correlation estimates for the period prior to the
global financial crisis. All country pairs display correlation with each
other of 60 percent or more, apart from with Korea. This low correlation
of the Korean bond returns with every country is indicative in figure
7.4, which shows that correlation with any country pair, rarely goes
beyond 30 percent. This provides for support for the argument that
global financial markets are becoming more integrated. Additionally, the
highest correlation, in excess of 80 percent, is between the Australian
and New Zealand pair and the US and Canadian pair. This supports the
evidence provided by Barr and Priestley
(\protect\hyperlink{ref-barr2004}{2004}), who finds high correlation
between the US and Canadian bond markets.

The Dynamic conditional correlation results do decrease by a significant
amount in some country pairs in the period after the global financial
crisis. However, the correlation is largely still above the 50 percent
mark. Figures \ref{figure9} to \ref{figure14} show that almost all
country pairs see a reduction in correlation in the post-crisis period.
The exception is the Canada and US pair, which figure \ref{figure10}
shows increases to almost 90 percent which provides further evidence to
the argument made by Barr and Priestley
(\protect\hyperlink{ref-barr2004}{2004}) and illustrates the importance
of diversification. Moreover, the correlation between Korea and all the
countries in the sample increases substantially in the post-crisis
period. This reaches as high as 60 percent between the Korean and
Australian pair. In general, the correlation between all the developed
countries is very high and demonstrates the need for diversification.
This can come in the form of alternative assets or even the inclusion of
developing economies to the portfolio.

On average, the bivariate correlation between the country pairs do
increase between the pre-crisis and post-crisis periods. Whilst
individual changes may have been large/small as mentioned above, the
correlation in general, shows a moderate increase between the two
periods. The only exception is that of Korea. The bivariate correlation
between Korea and the other 5 countries, prior to the crisis, increases
by approximately 20-25 percent in the post-crisis period. However, these
aggregate change interpretations are done merely by looking at the
correlation before and after the crisis. Aggregating the data for the
pre-crisis and post-crisis periods will yield more reliable
interpretations but will not be done in this paper due to time
constraints.

\newpage

\section{\texorpdfstring{Conclusion
\label{conclusion}}{Conclusion }}\label{conclusion}

In recapitulating the findings of this paper, it is evident that
correlation amongst the 6 developed economies is very high. This is
explained in part by the growing integration of the global financial
markets, but also by uncertainty and similar financial market conditions
in the countries selected. The Korean bond returns display the lowest
correlation with all the countries in the sample, while the pairs of
Canada and the US and the pairs of Australia and New Zealand display the
highest correlation. The high correlation in general, demonstrates the
need for diversification. Including other countries with different
financial market conditions and characteristics may yield the benefit of
diversification. Additionally, inclusion of alternative assets in the
portfolio, may yield diversification benefits as well.

It must be noted that this paper is limited in its evaluation of
correlation amongst developed market economies, in that few countries
are selected due to time constraints and data constraints. Future
studies can improve on this study, by including more countries and
applying less constraints to the DCC model.

\newpage

\section{\texorpdfstring{Appendix
\label{appendix}}{Appendix }}\label{appendix}

\begin{figure}[H]

{\centering \includegraphics{dian_finmetrics_project_files/figure-latex/figure3-1} 

}

\caption{Dynanmic correlation \label{figure3}}\label{fig:figure3}
\end{figure}

\begin{figure}[H]

{\centering \includegraphics{dian_finmetrics_project_files/figure-latex/figure4-1} 

}

\caption{Dynanmic correlation \label{figure4}}\label{fig:figure4}
\end{figure}

\begin{figure}[H]

{\centering \includegraphics{dian_finmetrics_project_files/figure-latex/figure5-1} 

}

\caption{Dynanmic correlation \label{figure5}}\label{fig:figure5}
\end{figure}

\begin{figure}[H]

{\centering \includegraphics{dian_finmetrics_project_files/figure-latex/figure6-1} 

}

\caption{Dynanmic correlation \label{figure6}}\label{fig:figure6}
\end{figure}

\begin{figure}[H]

{\centering \includegraphics{dian_finmetrics_project_files/figure-latex/figure7-1} 

}

\caption{Dynanmic correlation \label{figure7}}\label{fig:figure7}
\end{figure}

\begin{figure}[H]

{\centering \includegraphics{dian_finmetrics_project_files/figure-latex/figure8-1} 

}

\caption{Dynanmic correlation \label{figure8}}\label{fig:figure8}
\end{figure}

\begin{figure}[H]

{\centering \includegraphics{dian_finmetrics_project_files/figure-latex/figure9-1} 

}

\caption{Dynanmic correlation \label{figure9}}\label{fig:figure9}
\end{figure}

\begin{figure}[H]

{\centering \includegraphics{dian_finmetrics_project_files/figure-latex/figure10-1} 

}

\caption{Dynanmic correlation \label{figure10}}\label{fig:figure10}
\end{figure}

\begin{figure}[H]

{\centering \includegraphics{dian_finmetrics_project_files/figure-latex/figure11-1} 

}

\caption{Dynanmic correlation \label{figure11}}\label{fig:figure11}
\end{figure}

\begin{figure}[H]

{\centering \includegraphics{dian_finmetrics_project_files/figure-latex/figure12-1} 

}

\caption{Dynanmic correlation \label{figure12}}\label{fig:figure12}
\end{figure}

\begin{figure}[H]

{\centering \includegraphics{dian_finmetrics_project_files/figure-latex/figure13-1} 

}

\caption{Dynanmic correlation \label{figure13}}\label{fig:figure13}
\end{figure}

\begin{figure}[H]

{\centering \includegraphics{dian_finmetrics_project_files/figure-latex/figure14-1} 

}

\caption{Dynanmic correlation \label{figure14}}\label{fig:figure14}
\end{figure}

\newpage

\section*{\texorpdfstring{References
\label{references}}{References }}\label{references}
\addcontentsline{toc}{section}{References \label{references}}

\hypertarget{refs}{}
\hypertarget{ref-barr2004}{}
Barr, David G, and Richard Priestley. 2004. ``Expected Returns, Risk and
the Integration of International Bond Markets.'' \emph{Journal of
International Money and Finance} 23 (1). Elsevier: 71--97.

\hypertarget{ref-cappiello2006}{}
Cappiello, Lorenzo, Robert F Engle, and Kevin Sheppard. 2006.
``Asymmetric Dynamics in the Correlations of Global Equity and Bond
Returns.'' \emph{Journal of Financial Econometrics} 4 (4). Oxford
University Press: 537--72.

\hypertarget{ref-davies2007}{}
Davies, Andrew. 2007. ``International Bond Market Cointegration Using
Regime Switching Techniques.'' \emph{The Journal of Fixed Income} 16
(4). Euromoney Institutional Investor PLC: 69.

\hypertarget{ref-engle2002dynamic}{}
Engle, Robert. 2002. ``Dynamic Conditional Correlation: A Simple Class
of Multivariate Generalized Autoregressive Conditional
Heteroskedasticity Models.'' \emph{Journal of Business \& Economic
Statistics} 20 (3). Taylor \& Francis: 339--50.

\hypertarget{ref-FRENCH}{}
Fama, Eugene F, and Kenneth R French. 1989. ``Business Conditions and
Expected Returns on Stocks and Bonds.'' \emph{Journal of Financial
Economics} 25 (1). Elsevier: 23--49.

\hypertarget{ref-Texevier}{}
Katzke, N.F. 2017. \emph{Texevier: Package to Create Elsevier Templates
for Rmarkdown}. Stellenbosch, South Africa: Bureau for Economic
Research.

\hypertarget{ref-katzke2013}{}
Katzke, Nico. 2013. ``South African Sector Return Correlations: Using
Dcc and Adcc Multivariate Garch Techniques to Uncover the Underlying
Dynamics.'' \emph{South African Sector Return Correlations: Using DCC
and ADCC Multivariate GARCH Techniques to Uncover the Underlying
Dynamics}.

\hypertarget{ref-kumar2011}{}
Kumar, Manmohan S, and Tatsuyoshi Okimoto. 2011. ``Dynamics of
International Integration of Government Securities' Markets.''
\emph{Journal of Banking \& Finance} 35 (1). Elsevier: 142--54.

\hypertarget{ref-lucey2006}{}
Lucey, Brian M, and James Steeley. 2006. ``Measuring and Assessing the
Effects and Extent of International Bond Market Integration.''
\emph{Journal of International Financial Markets, Institutions and
Money} 16 (1). Elsevier: 1--3.

% Force include bibliography in my chosen format:

\bibliographystyle{Tex/Texevier}
\bibliography{Tex/ref}





\end{document}
